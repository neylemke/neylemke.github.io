% Created 2017-01-29 Sun 00:40
% Intended LaTeX compiler: pdflatex
\documentclass[11pt]{article}
\usepackage[utf8]{inputenc}
\usepackage[T1]{fontenc}
\usepackage{graphicx}
\usepackage{grffile}
\usepackage{longtable}
\usepackage{wrapfig}
\usepackage{rotating}
\usepackage[normalem]{ulem}
\usepackage{amsmath}
\usepackage{textcomp}
\usepackage{amssymb}
\usepackage{capt-of}
\usepackage{hyperref}
\author{Ney Lemke}
\date{\today}
\title{Examplo de Uso do Reveal no Orgmode}
\hypersetup{
 pdfauthor={Ney Lemke},
 pdftitle={Examplo de Uso do Reveal no Orgmode},
 pdfkeywords={},
 pdfsubject={},
 pdfcreator={Emacs 24.5.1 (Org mode 9.0.4)}, 
 pdflang={English}}
\begin{document}

\maketitle
\setcounter{tocdepth}{1}
\tableofcontents


\section*{Geração Automática de Slides}
\label{sec:org3b475a4}

\subsection*{Objetivo}
\label{sec:org2448c48}
O meu objetivo inicial vai ser testar se é mais eficiente gerar 
slides usando o orgmode. 

\subsection*{Inicialmente vamos preparar o sistema}
\label{sec:org67cff91}
\begin{verbatim}
git clone https://github.com/ipython-books/cookbook-data.git
cd cookbook-data
unzip tennis.zip
\end{verbatim}

\subsection*{Vamos testar o uso do python}
\label{sec:orga1d3243}
Os primeiros passos são carregar os pacotes:

\begin{verbatim}
import numpy as np
import pandas as pd
import matplotlib.pyplot as plt
\end{verbatim}

\subsection*{Datafile}
\label{sec:orgb404e8a}

Criamos o \emph{datafile} usando os arquivos do item anterior. 

\begin{verbatim}
player = 'Roger Federer'
filename = "./cookbook-data/data/{name}.csv".format(name=player.replace(' ', '-')) 
df = pd.read_csv(filename)
\end{verbatim}

\subsection*{Determinamos quem Venceu}
\label{sec:orgefba962}
\begin{verbatim}
df['win'] = df['winner'] == player
df['win'].tail()
\end{verbatim}

\begin{verbatim}
1174    False
1175     True
1176     True
1177     True
1178    False
Name: win, dtype: bool
\end{verbatim}

\subsection*{Podemos gerar um \emph{Report} mais interessante.}
\label{sec:org4487018}

\begin{verbatim}
("{player} has won {vic:.0f}% "
               "of his ATP matches.").format(
                player=player, vic=100*df['win'].mean())
\end{verbatim}


\subsection*{Agora geramos a fração de faltas duplas}
\label{sec:org55cec50}
\begin{verbatim}
df['dblfaults'] = (df['player1 double faults'] / 
                   df['player1 total points total'])
\end{verbatim}

\begin{verbatim}
df['dblfaults'].tail()
\end{verbatim}

\subsection*{Agrupamos os dados}
\label{sec:orga292ab8}

Inicialmente por tipo de quadra
\begin{verbatim}
df.groupby('surface')['win'].mean()
\end{verbatim}

\subsection*{Agrupamos os dados}
\label{sec:org9c41ce5}

Agora por ano 

\begin{verbatim}
gb = df.groupby('year')
\end{verbatim}


\subsection*{Gráfico dos Pontos}
\label{sec:org0cf5575}
\begin{verbatim}
%matplotlib inline
fig=plt.figure(figsize=(4,2))
plt.plot_date(gb['start date'].max(), gb['dblfaults'].mean(), '-', lw=3,tz='UTC')
plt.plot_date(df['start date'], df['dblfaults'], alpha=.25, lw=0,tz='UTC')
plt.ylabel('Proportion of double faults per match.')
plt.xlabel('Year')
plt.savefig('images/pandas.png')
\end{verbatim}


\subsection*{Melhores Resultados}
\label{sec:orgd0e3f7d}
\begin{verbatim}
gb['start date'].max()
\end{verbatim}
\end{document}